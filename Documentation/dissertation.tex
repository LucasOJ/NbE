% The document class supplies options to control rendering of some standard
% features in the result.  The goal is for uniform style, so some attention 
% to detail is *vital* with all fields.  Each field (i.e., text inside the
% curly braces below, so the MEng text inside {MEng} for instance) should 
% take into account the following:
%
% - author name       should be formatted as "FirstName LastName"
%   (not "Initial LastName" for example),
% - supervisor name   should be formatted as "Title FirstName LastName"
%   (where Title is "Dr." or "Prof." for example),
% - degree programme  should be "BSc", "MEng", "MSci", "MSc" or "PhD",
% - dissertation title should be correctly capitalised (plus you can have
%   an optional sub-title if appropriate, or leave this field blank),
% - dissertation type should be formatted as one of the following:
%   * for the MEng degree programme either "enterprise" or "research" to
%     reflect the stream,
%   * for the MSc  degree programme "$X/Y/Z$" for a project deemed to be
%     X%, Y% and Z% of type I, II and III.
% - year              should be formatted as a 4-digit year of submission
%   (so 2014 rather than the academic year, say 2013/14 say).

\documentclass[ % the name of the author
                    author={Lucas O'Dowd-Jones},
                % the name of the supervisor
                supervisor={Dr. Alex Kavvos},
                % the degree programme
                    degree={MEng},
                % the dissertation    title (which cannot be blank)
                     title={Variations on Normalisation by Evaluation in Haskell},
                % the dissertation subtitle (which can    be blank)
                  subtitle={},
                % the dissertation     type
                      type={programming languages},
                % the year of submission
                      year={2021} ]{dissertation}

\usepackage{amsmath}
\usepackage{biblatex}
\addbibresource{bibliography.bib}

\begin{document}

% =============================================================================

% This macro creates the standard UoB title page by using information drawn
% from the document class (meaning it is vital you select the correct degree 
% title and so on).

\maketitle

% After the title page (which is a special case in that it is not numbered)
% comes the front matter or preliminaries; this macro signals the start of
% such content, meaning the pages are numbered with Roman numerals.

\frontmatter

% This macro creates the standard UoB declaration; on the printed hard-copy,
% this must be physically signed by the author in the space indicated.

\makedecl

% LaTeX automatically generates a table of contents, plus associated lists 
% of figures, tables and algorithms.  The former is a compulsory part of the
% dissertation, but if you do not require the latter they can be suppressed
% by simply commenting out the associated macro.

\tableofcontents
\listoffigures
\listoftables
\listofalgorithms
\lstlistoflistings

% The following sections are part of the front matter, but are not generated
% automatically by LaTeX; the use of \chapter* means they are not numbered.

% -----------------------------------------------------------------------------

\chapter*{Executive Summary}

% {\bf A compulsory section, of at most $1$ page} 

\noindent
This section should pr\'{e}cis the project context, aims and objectives,
and main contributions (e.g., deliverables) and achievements; the same 
section may be called an abstract elsewhere.  The goal is to ensure the 
reader is clear about what the topic is, what you have done within this 
topic, {\em and} what your view of the outcome is.

The former aspects should be guided by your specification: essentially 
this section is a (very) short version of what is typically the first 
chapter.  Note that for research-type projects, this {\bf must} include 
a clear research hypothesis.  This will obviously differ significantly
for each project, but an example might be as follows:

\begin{quote}
My research hypothesis is that a suitable genetic algorithm will yield
more accurate results (when applied to the standard ACME data set) than 
the algorithm proposed by Jones and Smith, while also executing in less
time.
\end{quote}

\noindent
The latter aspects should (ideally) be presented as a concise, factual 
bullet point list.  Again the points will differ for each project, but 
an might be as follows:

\begin{quote}
\noindent
\begin{itemize}
\item I spent $120$ hours collecting material on and learning about the 
      Java garbage-collection sub-system. 
\item I wrote a total of $5000$ lines of source code, comprising a Linux 
      device driver for a robot (in C) and a GUI (in Java) that is 
      used to control it.
\item I designed a new algorithm for computing the non-linear mapping 
      from A-space to B-space using a genetic algorithm, see page $17$.
\item I implemented a version of the algorithm proposed by Jones and 
      Smith in [6], see page $12$, corrected a mistake in it, and 
      compared the results with several alternatives.
\end{itemize}
\end{quote}

% -----------------------------------------------------------------------------

\chapter*{Supporting Technologies}

% {\bf A compulsory section, of at most $1$ page}

\noindent
This section should present a detailed summary, in bullet point form, 
of any third-party resources (e.g., hardware and software components) 
used during the project.  Use of such resources is always perfectly 
acceptable: the goal of this section is simply to be clear about how
and where they are used, so that a clear assessment of your work can
result.  The content can focus on the project topic itself (rather,
for example, than including ``I used \mbox{\LaTeX} to prepare my 
dissertation''); an example is as follows:

\begin{quote}
\noindent
\begin{itemize}
\item I used the Java {\tt BigInteger} class to support my implementation 
      of RSA.
\item I used a parts of the OpenCV computer vision library to capture 
      images from a camera, and for various standard operations (e.g., 
      threshold, edge detection).
\item I used an FPGA device supplied by the Department, and altered it 
      to support an open-source UART core obtained from 
      \url{http://opencores.org/}.
\item The web-interface component of my system was implemented by 
      extending the open-source WordPress software available from
      \url{http://wordpress.org/}.
\end{itemize}
\end{quote}

% -----------------------------------------------------------------------------

\chapter*{Notation and Acronyms}

% {\bf An optional section, of roughly $1$ or $2$ pages}

\noindent
Any well written document will introduce notation and acronyms before
their use, {\em even if} they are standard in some way: this ensures 
any reader can understand the resulting self-contained content.  

Said introduction can exist within the dissertation itself, wherever 
that is appropriate.  For an acronym, this is typically achieved at 
the first point of use via ``Advanced Encryption Standard (AES)'' or 
similar, noting the capitalisation of relevant letters.  However, it 
can be useful to include an additional, dedicated list at the start 
of the dissertation; the advantage of doing so is that you cannot 
mistakenly use an acronym before defining it.  A limited example is 
as follows:

\begin{quote}
\noindent
\begin{tabular}{lcl}
AES                 &:     & Advanced Encryption Standard                                         \\
DES                 &:     & Data Encryption Standard                                             \\
                    &\vdots&                                                                      \\
${\mathcal H}( x )$ &:     & the Hamming weight of $x$                                            \\
${\mathbb  F}_q$    &:     & a finite field with $q$ elements                                     \\
$x_i$               &:     & the $i$-th bit of some binary sequence $x$, st. $x_i \in \{ 0, 1 \}$ \\
\end{tabular}
\end{quote}

% -----------------------------------------------------------------------------

\chapter*{Acknowledgements}

% {\bf An optional section, of at most $1$ page}

\noindent
It is common practice (although totally optional) to acknowledge any
third-party advice, contribution or influence you have found useful
during your work.  Examples include support from friends or family, 
the input of your Supervisor and/or Advisor, external organisations 
or persons who  have supplied resources of some kind (e.g., funding, 
advice or time), and so on.

% =============================================================================

% After the front matter comes a number of chapters; under each chapter,
% sections, subsections and even subsubsections are permissible.  The
% pages in this part are numbered with Arabic numerals.  Note that:
%
% - A reference point can be marked using \label{XXX}, and then later
%   referred to via \ref{XXX}; for example Chapter\ref{chap:context}.
% - The chapters are presented here in one file; this can become hard
%   to manage.  An alternative is to save the content in seprate files
%   the use \input{XXX} to import it, which acts like the #include
%   directive in C.

\mainmatter

% -----------------------------------------------------------------------------

\chapter{Introduction}
\label{chap:introduction}

% Start with want to implement functional language
To implement a functional programming language, we need a normalisation function that maps each expression in the functional language to its normal form.
% We use NbE
In this dissertation we explore various implementations of Normalisation by Evaluation (NbE) for the lambda calculus.

% What/how it works roughly
NbE proceeds by interpreting terms of the lambda calculus (referred to as “syntax”) as elements of a mathematical “semantic” set. NbE then “reifies” semantic values back into the set of normal terms. All $\beta$-equal terms “evaluate” to the same semantic value, so $\beta$-equal terms normalise to the same normal form.

% Why do we care
    % Modern
NbE is a modern alternative to normalisation by reduction; a technique based on syntactic rewriting. NbE is useful for the following reasons.

\begin{enumerate}
    \item Since the foundations of NbE are mathematical, we can prove that our implementation is correct and study its behaviour formally \cite{AgdaNbe}. Proving that the implementations are fully correct is beyond the scope of this project, but we use types as machine-checked proof that our implementation satisfies certain properties.
    \item Some research suggests that NbE can improve the speed of compilation of functional languages \cite{efficientNbE}.
    \item Research is ongoing into whether dependent type theories such as Coq can use NbE to check for equality between dependent types by normalising type-level programs \cite{deBruijn}
\end{enumerate}

% --roadmap
To become familiar with the general operation of NbE, in Chapter \ref{chap:untypednbe} we present two implementations for NbE of the untyped lambda calculus. The first implementation normalises the named lambda calculus. This implementation works, but its correctness depends on fresh name generation which makes it difficult to reason about. This issue motivates a second implementation of NbE, where we normalise a nameless representation of the lambda terms instead. Since variable names are not part of this syntax, fresh variables are much easier to generate.

Then we move to the central challenge of this project: translating an Agda implementation of NbE for the Simply Typed Lambda Calculus (STLC) into Haskell. The Agda implementation we follow makes use of advanced type-level features such as dependent types and dependent pattern matching. However, Agda is currently unsuitable for general-purpose programming for the following reasons

\begin{enumerate}
    \item It is a type theory and proof assistant primarily, rather than a general-purpose programming language
    \item It has a steep learning curve due to the theoretical understanding required to develop programs
    \item The community supporting it is small and mainly academic
\end{enumerate}

Haskell is a mature language suitable for industry use \cite{haskellInIndustry} as it strikes a balance between theory and practice, allowing developers to take advantage of type-safety without as much theoretical overhead. However, recent compiler options have enabled the use of advanced type-features more akin to dependently typed languages such as Agda. 

Through implementing NbE for the STLC, this project will assess whether Haskell can be used to emulate features of languages with full dependent types. Members of the Haskell community are actively working on bringing full dependent types to Haskell \cite{DH}, but this project serves as an evaluation of how well the existing tooling for complex types works in practice.

In Chapter \ref{chap:typedlamadacalculus}, we use introduce and use Generalised Algebraic Datatypes (GADTs) in conjunction with the \code{DataKinds} and \code{PolyKinds} extensions to define terms that are well-typed by construction. 
In Chapter \ref{chap:typednbe} we implement the well-typed normalisation function. For this we need the \code{RankNTypes} extension, which gives finer control over quantification in polymorphic type signatures, and \code{ScopedTypeVariables}, to bind type variables within function bodies. To emulate reflection of dependent types from the type level to the value level at runtime, we explore a method inspired by the singleton pattern \cite{singletons}.

The aims of this project are:
\begin{enumerate}
    \item To produce various implementations of NbE in Haskell
    \item To explore how successful modern features of Haskell are in implementing an algorithm with complex types.
\end{enumerate}

% -----------------------------------------------------------------------------

\chapter{The Structure of Normalisation by Evaluation}
\label{chap:nbestructure}

% -----------------------------------------------------------------------------

\chapter{Normalising the Untyped Lambda Calculus}
\label{chap:untypednbe}

% -----------------------------------------------------------------------------

\chapter{Representation of the Typed Lambda Calculus}
\label{chap:typedlamadacalculus}

% -----------------------------------------------------------------------------

\chapter{Normalising the Typed Lambda Calculus}
\label{chap:typednbe}

% -----------------------------------------------------------------------------

\chapter{Critical Evaluation}
\label{chap:evaluation}

%% {\bf A topic-specific chapter, of roughly $15$ pages} 

\noindent
This chapter is intended to evaluate what you did.  The content is highly 
topic-specific, but for many projects will have flavours of the following:

\begin{enumerate}
\item functional  testing, including analysis and explanation of failure 
      cases,
\item behavioural testing, often including analysis of any results that 
      draw some form of conclusion wrt. the aims and objectives,
      and
\item evaluation of options and decisions within the project, and/or a
      comparison with alternatives.
\end{enumerate}

\noindent
This chapter often acts to differentiate project quality: even if the work
completed is of a high technical quality, critical yet objective evaluation 
and comparison of the outcomes is crucial.  In essence, the reader wants to
learn something, so the worst examples amount to simple statements of fact 
(e.g., ``graph X shows the result is Y''); the best examples are analytical 
and exploratory (e.g., ``graph X shows the result is Y, which means Z; this 
contradicts [1], which may be because I use a different assumption'').  As 
such, both positive {\em and} negative outcomes are valid {\em if} presented 
in a suitable manner.

% -----------------------------------------------------------------------------

\chapter{Conclusion}
\label{chap:conclusion}

% {\bf A compulsory chapter,     of roughly $5$ pages} 

\noindent
The concluding chapter of a dissertation is often underutilised because it 
is too often left too close to the deadline: it is important to allocation
enough attention.  Ideally, the chapter will consist of three parts:

\begin{enumerate}
\item (Re)summarise the main contributions and achievements, in essence
      summing up the content.
\item Clearly state the current project status (e.g., ``X is working, Y 
      is not'') and evaluate what has been achieved with respect to the 
      initial aims and objectives (e.g., ``I completed aim X outlined 
      previously, the evidence for this is within Chapter Y'').  There 
      is no problem including aims which were not completed, but it is 
      important to evaluate and/or justify why this is the case.
\item Outline any open problems or future plans.  Rather than treat this
      only as an exercise in what you {\em could} have done given more 
      time, try to focus on any unexplored options or interesting outcomes
      (e.g., ``my experiment for X gave counter-intuitive results, this 
      could be because Y and would form an interesting area for further 
      study'' or ``users found feature Z of my software difficult to use,
      which is obvious in hindsight but not during at design stage; to 
      resolve this, I could clearly apply the technique of Smith [7]'').
\end{enumerate}

% =============================================================================

% Finally, after the main matter, the back matter is specified.  This is
% typically populated with just the bibliography.  LaTeX deals with these
% in one of two ways, namely
%
% - inline, which roughly means the author specifies entries using the 
%   \bibitem macro and typesets them manually, or
% - using BiBTeX, which means entries are contained in a separate file
%   (which is essentially a databased) then inported; this is the 
%   approach used below, with the databased being dissertation.bib.
%
% Either way, the each entry has a key (or identifier) which can be used
% in the main matter to cite it, e.g., \cite{X}, \cite[Chapter 2}{Y}.

\backmatter

\printbibliography

% -----------------------------------------------------------------------------

% The dissertation concludes with a set of (optional) appendicies; these are 
% the same as chapters in a sense, but once signaled as being appendicies via
% the associated macro, LaTeX manages them appropriatly.

\appendix

\chapter{An Example Appendix}
\label{appx:example}

Content which is not central to, but may enhance the dissertation can be 
included in one or more appendicies.

\noindent
Note that in line with most research conferences, the marking panel is not
obliged to read such appendices.

% =============================================================================

\end{document}
