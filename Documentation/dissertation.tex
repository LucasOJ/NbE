% The document class supplies options to control rendering of some standard
% features in the result.  The goal is for uniform style, so some attention 
% to detail is *vital* with all fields.  Each field (i.e., text inside the
% curly braces below, so the MEng text inside {MEng} for instance) should 
% take into account the following:
%
% - author name       should be formatted as "FirstName LastName"
%   (not "Initial LastName" for example),
% - supervisor name   should be formatted as "Title FirstName LastName"
%   (where Title is "Dr." or "Prof." for example),
% - degree programme  should be "BSc", "MEng", "MSci", "MSc" or "PhD",
% - dissertation title should be correctly capitalised (plus you can have
%   an optional sub-title if appropriate, or leave this field blank),
% - dissertation type should be formatted as one of the following:
%   * for the MEng degree programme either "enterprise" or "research" to
%     reflect the stream,
%   * for the MSc  degree programme "$X/Y/Z$" for a project deemed to be
%     X%, Y% and Z% of type I, II and III.
% - year              should be formatted as a 4-digit year of submission
%   (so 2014 rather than the academic year, say 2013/14 say).

\documentclass[ % the name of the author
                    author={Lucas O'Dowd-Jones},
                % the name of the supervisor
                supervisor={Dr. Alex Kavvos},
                % the degree programme
                    degree={MEng},
                % the dissertation    title (which cannot be blank)
                     title={Variations on Normalisation by Evaluation in Haskell},
                % the dissertation subtitle (which can    be blank)
                  subtitle={},
                % the dissertation     type
                      type={programming languages},
                % the year of submission
                      year={2021}]{dissertation}

\usepackage{amsmath}
\usepackage{biblatex}
\usepackage{stmaryrd}
\addbibresource{bibliography.bib}
\usepackage{listings}
\lstset{
  basicstyle=\ttfamily\small,
  keywordstyle=[0]\bfseries\color{bittersweet},
  keywordstyle=[1]\bfseries\color{teal},
  keywordstyle=[3]\bfseries\color{dark},
  keywordstyle=[2]\bfseries,
  columns=spaceflexible,
  morekeywords=[0]{data, where, type, class, instance, case, of, do, return, infixr},
  morekeywords=[1]{Name, String, Expr, NormalForm, NeutralForm, Env, Map, V, NeutralV, FreshName, State, Set, DbExpr, Int, Ty, Nat, Vec, Elem, OPE, NeutralExpr, NormalExpr, forall, SingTy, STy, Any, SingContext, 'BaseTy, Eq, ChurchNumeralTy, ChurchNumeral, NFData, [Name], [Ty]},
  morekeywords=[2]{Z, S, True, case, of},
  morekeywords=[3]{ctx, ty, arg, result, ctx1, ctx2, strong, weak, ctxV, ctx', ctxV', strongest, a, b, c, arg1, arg2},
  mathescape=true,
  keepspaces=true,
  aboveskip=10pt,
  belowskip=10pt,
  alsoletter={', [, ]}
}

\newcommand{\code}{\lstinline}

\begin{document}
\definecolor{teal}{rgb}{0.0, 0.5, 0.5}
\definecolor{bittersweet}{rgb}{1.0, 0.44, 0.37}
\definecolor{dark}{rgb}{0.1, 0.45, 1.0}

% =============================================================================

% This macro creates the standard UoB title page by using information drawn
% from the document class (meaning it is vital you select the correct degree 
% title and so on).

\maketitle

% After the title page (which is a special case in that it is not numbered)
% comes the front matter or preliminaries; this macro signals the start of
% such content, meaning the pages are numbered with Roman numerals.

\frontmatter

% This macro creates the standard UoB declaration; on the printed hard-copy,
% this must be physically signed by the author in the space indicated.

\makedecl

% LaTeX automatically generates a table of contents, plus associated lists 
% of figures, tables and algorithms.  The former is a compulsory part of the
% dissertation, but if you do not require the latter they can be suppressed
% by simply commenting out the associated macro.

\tableofcontents

% \listoffigures
% \listoftables
% \listofalgorithms
% \lstlistoflistings

% The following sections are part of the front matter, but are not generated
% automatically by LaTeX; the use of \chapter* means they are not numbered.

% -----------------------------------------------------------------------------

\chapter*{Executive Summary}

% {\bf A compulsory section, of at most $1$ page} 

\noindent
This section should pr\'{e}cis the project context, aims and objectives,
and main contributions (e.g., deliverables) and achievements; the same 
section may be called an abstract elsewhere.  The goal is to ensure the 
reader is clear about what the topic is, what you have done within this 
topic, {\em and} what your view of the outcome is.

The former aspects should be guided by your specification: essentially 
this section is a (very) short version of what is typically the first 
chapter.  Note that for research-type projects, this {\bf must} include 
a clear research hypothesis.  This will obviously differ significantly
for each project, but an example might be as follows:

\begin{quote}
My research hypothesis is that a suitable genetic algorithm will yield
more accurate results (when applied to the standard ACME data set) than 
the algorithm proposed by Jones and Smith, while also executing in less
time.
\end{quote}

\noindent
The latter aspects should (ideally) be presented as a concise, factual 
bullet point list.  Again the points will differ for each project, but 
an might be as follows:

\begin{quote}
\noindent
\begin{itemize}
\item I spent $120$ hours collecting material on and learning about the 
      Java garbage-collection sub-system. 
\item I wrote a total of $5000$ lines of source code, comprising a Linux 
      device driver for a robot (in C) and a GUI (in Java) that is 
      used to control it.
\item I designed a new algorithm for computing the non-linear mapping 
      from A-space to B-space using a genetic algorithm, see page $17$.
\item I implemented a version of the algorithm proposed by Jones and 
      Smith in [6], see page $12$, corrected a mistake in it, and 
      compared the results with several alternatives.
\end{itemize}
\end{quote}

% -----------------------------------------------------------------------------

\chapter*{Acknowledgements}

% {\bf An optional section, of at most $1$ page}

\noindent
It is common practice (although totally optional) to acknowledge any
third-party advice, contribution or influence you have found useful
during your work.  Examples include support from friends or family, 
the input of your Supervisor and/or Advisor, external organisations 
or persons who  have supplied resources of some kind (e.g., funding, 
advice or time), and so on.

% =============================================================================

% After the front matter comes a number of chapters; under each chapter,
% sections, subsections and even subsubsections are permissible.  The
% pages in this part are numbered with Arabic numerals.  Note that:
%
% - A reference point can be marked using \label{XXX}, and then later
%   referred to via \ref{XXX}; for example Chapter\ref{chap:context}.
% - The chapters are presented here in one file; this can become hard
%   to manage.  An alternative is to save the content in seprate files
%   the use \input{XXX} to import it, which acts like the #include
%   directive in C.

\mainmatter

% -----------------------------------------------------------------------------

\chapter{Introduction}
\label{chap:introduction}

% Start with want to implement functional language
To implement a functional programming language, we need a normalisation function that maps each expression in the functional language to its normal form.
% We use NbE
In this dissertation we explore various implementations of normalisation by evaluation (NbE) for the lambda calculus.

% What/how it works roughly
NbE proceeds by interpreting terms of the lambda calculus (referred to as “syntax”) as elements of a mathematical “semantic” set, where computation is easier to perform. NbE then “reifies” semantic values back into the set of normal terms. All $\beta$-equal terms “evaluate” to the same semantic value, so $\beta$-equal terms normalise to the same normal form.

% Why do we care
    % Modern
NbE is a modern alternative to normalisation by reduction; a technique based on syntactic rewriting. NbE is useful for the following reasons.

\begin{enumerate}
    \item Since the foundations of NbE are mathematical, we can prove that our implementation is correct and study its behaviour formally \cite{AgdaNbe}. Proving that the implementations are fully correct is beyond the scope of this project, but we use types as machine-checked proof that our implementation satisfies certain properties.
    \item Some research suggests that NbE can improve the speed of compilation of functional languages \cite{efficientNbE}.
    \item Research is ongoing into whether dependent type theories such as Coq can use NbE to check for equality between dependent types by normalising type-level programs 
\end{enumerate}

% --roadmap
To become familiar with the general operation of NbE, in Chapter \ref{chap:untypednbe} we present two implementations for NbE of the untyped lambda calculus. The first implementation normalises the named lambda calculus. This implementation works but its correctness depends on fresh name generation which makes it difficult to reason about. This issue motivates a second implementation of NbE, where we normalise a nameless representation of the lambda terms instead. Since variable names are not part of this syntax, fresh variables but much easier to generate.

Then we move to the central challenge of this project: translating an Agda implementation of NbE for the Simply Typed Lambda Calculus (STLC) into Haskell. The Agda implementation we follow makes use of advanced type-level features such as dependent types and dependent pattern matching. However, Agda is currently unsuitable for industrial use for the following reasons

\begin{enumerate}
    \item It is a type theory and proof assistant primarily, rather than a general purpose programming language
    \item It has a steep learning curve due to the theoretical understanding required to develop programs
    \item Compilation is slow [CITE]
    \item The community supporting it is small and mainly academic
\end{enumerate}

Haskell is a mature language suitable for industry use [CITE HASKELL INDUSTRY USAGE] as it strikes a balance between theory and practice, allowing developers to take advantage of type-safety without as much theoretical overhead. However, recent compiler options have enabled the use of advanced type-features more akin to dependently typed languages. 

Through implementing NbE for the STLC, this project will assess whether Haskell can be used to emulate features of languages with full dependent types. Members of the Haskell community are actively working on bringing full dependent types to Haskell \cite{DH}, but this project serves as an evaluation of how well the existing tooling for complex types work in practice.

In Chapter \ref{chap:typedlamadacalculus}, we use introduce and use Generalised Algebraic Datatypes (GADTs) in conjunction with the \code{DataKinds} and \code{PolyKinds} extensions to define terms that are well-typed by construction. 

In Chapter \ref{chap:typednbe} we implement the well-typed normalisation function. For this we need the \code{RankNTypes} extension, which gives finer control over quantification in polymorphic type signatures, and \code{ScopedTypeVariables}, to bind type variables within function bodies. To emulate reflection of dependent types from the type level to the value level at runtime we explore the singleton pattern.

The aims of this project are:
\begin{enumerate}
    \item To produce various implementations of NbE in Haskell
    \item To explore how successful modern features of Haskell are in implementing an algorithm with complex types.
\end{enumerate}

% -----------------------------------------------------------------------------

\chapter{Normalising the Untyped Lambda Calculus}
\label{chap:untypednbe}

\section{Syntax}

\begin{lstlisting}
    type Name = String

    data Expr = ExpVar Name
              | ExpLam Name Expr
              | ExpApp Expr Expr
\end{lstlisting}

Inhabitants of the inductively-defined datatype \lstinline{Expr} are well-formed terms of the untyped lambda calculus with strings as variables. The first argument of the lambda case introduces a new variable name bound in the function body defined by the second argument. For example, the identity function $\lambda x . x$ would be encoded as \lstinline{ExpLam "x" (ExpVar "x")}. The set of terms defined by \lstinline{Expr} is the domain of the normalisation function.

\begin{lstlisting}
    data NormalForm = NfNeutralForm NeutralForm
                    | NfLam Name NormalForm

    data NeutralForm = NeVar Name
                     | NeApp NeutralForm NormalForm
\end{lstlisting}

We now define the target syntax \lstinline{NormalForm}; the codomain of the normalisation function. Note that \lstinline{NormalForm} is inhabited by all the terms not containing $\beta$-redexes \cite{slides}, since the definition of \lstinline{NeApp} only permits the application of non-lambda terms, which are encoded as values of type \lstinline{NeutralForm}.

%% Haskell useful - clean structully inductive defn even when mutualy-recursive

\section{Semantics}

The first step of NbE is evaluating a syntactic term into a semantic set \lstinline{V}, where all $\beta$-equal terms evaluate to the same element of the semantic set. [CITE]

\begin{lstlisting}
    data V = Neutral NeutralV
           | Function (V -> V)

    data NeutralV = NeVVar Name
                  | NeVApp NeutralV V
\end{lstlisting}

Could use Neutral Form but we establish our own datatype
Problem: fresh variables

\section{Evaluation}
To keep track of which variables have been bound by lambda terms, we introduce an environment to evaluate terms in.

\begin{lstlisting}
    type Env = Map Name V
\end{lstlisting}

Each key of the map corresponds to a bound variable name, and its associated value is the element of the semantic set representing the variable. 

\begin{lstlisting}
    eval :: Expr -> Env -> V
    eval (ExpVar x) env = case lookup x env of
            Just y -> y
            Nothing -> Neutral (NeVVar x)

    eval (ExpLam var m) env = Function f where
        f v = eval m env' where
            env' = insert var v env

    eval (ExpApp m n) env = app (eval m env) (eval n env)
\end{lstlisting}

The evaluation function takes an expression and the environment to evaluate it in, and returns the interpretation in the semantic set. 

In the variable case, we lookup the variable in the environment. If the variable was bound by an outer lambda term [WHAT IS AN OUTER LAMBDA TERM?], the variable will be present in the environment, and we can return the semantic value associated with it. Otherwise, the variables is free, so we return a semantic variable of the same name.

The semantic interpretation of a lambda expression is a Haskell function of type \lstinline{V} $\rightarrow$ \lstinline{V}. This function takes an element of the semantic set \lstinline{V}, and returns the body of the lambda evaluated in an extended environment \lstinline{env'}. In this modified environment, we introduce the variable \lstinline{var} bound by the lambda, and map this variable to . This construction is chosen as applying \lstinline{f} to an argument $v_0$ has the effect of substituting $v_0$ in place of the variable \lstinline{var} in the body of the lambda. This is similar to contracting a redex, however in NbE application of semantic elements replaces syntactic substitution.

\begin{lstlisting}
    app :: V -> V -> V
    app (Function f) v = f v
    app (Neutral n)  v = Neutral (NeVApp n v)
\end{lstlisting}


\begin{lstlisting}
    reify :: V -> FreshName NormalForm
    reify (Neutral n)  = do 
        reifiedN <- reifyNeutral n
        return (NfNeutralForm reifiedN)
    reify (Function f) = do
        freshNames <- get
        -- Remove the first name from the freshNames stream
        let freshVar = head freshNames
        -- The first name is no longer fresh (we are abount to use it as a bound variable)
        -- Modify the state to remove the used variable name
        modify tail
        -- Reify the body of the semantic function when evaluated at the fresh bound variable
        normalForm <- reify (f (Neutral (NeVVar freshVar)))
        return (NfLam freshVar normalForm)

    reifyNeutral :: NeutralV -> FreshName NeutralForm
    reifyNeutral (NeVVar i)   = return (NeVar i)
    reifyNeutral (NeVApp n m) = do
        reifiedNeutral <- reifyNeutral n
        reifiedNormal  <- reify m
        return (NeApp reifiedNeutral reifiedNormal)
\end{lstlisting}


\begin{lstlisting}
    normalise :: Expr -> NormalForm
    normalise exp = evalState (reify (eval exp empty)) freshNames 
        where
            freshNames = (getFreshVariableStream . getFreeVariables) exp
\end{lstlisting}

\section{Gensym Reification}

\subsection{Fresh variables solution 1 - Gensym}
approach based on \cite{slides}

Implemented with State monad

Issue with solution 1: Have to add monad everywhere (inescapable) - all functions dependent on state

"Less functional" - carry around state (may as well use imperative)
\section{de Bruijn Reification}
\subsection{Fresh variables solution 2 - Locally nameless terms}
approach based on \cite{deBruijn}

Uses de Bruijn Indicies for syntax and deBruijn levels for semantics

index n references nth abstraction,
if m abstractions: if n < m bound variables, otherwise free variable

Shifting for abstractions


% -----------------------------------------------------------------------------

\chapter{Representation of the Simply Typed Lambda Calculus}
\label{chap:typedlamadacalculus}

% TODO: STLC Abbreviation defininition
% TODO: Why choose STLC? Just as powerful as Turing Machine, simple

Before implementing NbE for the STLC, we define datatypes representing the typed lambda calculus in Haskell. 

\section{Types}

We first define a type syntax to represent the monotypes of the simply typed lambda calculus. 

\begin{lstlisting}
    data Ty = BaseTy | Ty :-> Ty 
    infixr 9 :->
\end{lstlisting}

In this type syntax there is a single base type \code{BaseTy} and an infix type constructor \code{:->}, where the type \code{A :-> B} represents the function type from type \code{A} to type \code{B}. For this implementation of NbE, a single base type is sufficient to capture the structure of simply typed terms. Multiple base types would have introduced unnecessary complexity.
%TODO: Justify why, type checking?
Normalising terms with polymorphic types is beyond the scope of this project.
 
\code{infixr :-> 9} specifies that \code{:->} is right associative, so as per convention, the type \code{A :-> (B :-> C)} can instead be written \code{A :-> B :-> C}.

\section{Typed Terms}

For typed NbE, we only normalise well-typed terms of the lambda calculus. Although well-typed terms are a subset of the untyped terms, the definition of untyped terms from Chapter \ref{chap:untypednbe} doesn't carry enough information to check that terms are well-typed. 
% TODO: Review, not great angle as type more related than just information storage
Instead, we define a richer definiton of typed terms that carry type information proposed by Richard Eisenberg \cite{GADTs}. This information will be stored in the Haskell type of each term value by utilising compiler extensions for GHC. 
% Well-typed by contruction better for haskell than type checking 

\begin{lstlisting}
    data Expr = Var Ty [Ty] Int
              | Lam Ty [Ty] Expr
              | App Ty [Ty] Expr Expr
\end{lstlisting}

% TODO: Repeated parameter - property of Expr' rather than constructor

\subsection{Introduction to Generalised Algebraic Datatypes (GADTs)}

GADTs are a generalisation of algebraic datatypes, where the type signature of each constructor is explicitly specified. The canonical example for the use of GADTs is length-indexed vectors, where the length of each vector is tracked in its type.

To track the length of the vector in its type, we first need a way of representing numbers at the type-level. A standard way of representing the natural numbers at value-level is as follows:

\begin{lstlisting}
    data Nat = Zero | Succ Nat
\end{lstlisting}

We use the \code{DataKinds} compiler extension to automatically create a kind \code{Nat}, with the same structure as the original \code{Nat} datatype. Thus, the \code{Nat} kind has the inhabitants \code{'Zero} and \code{'Succ}, which are denoted as types with apostrophes to prevent ambiguity with their value-level counterparts. These inhabitants are type constructors, where \code{'Zero} has kind \code{Nat}, and \code{'SuccNat} has kind \code{Nat} $\rightarrow$ \code{Nat}. 

\begin{figure}[h]
    \centering
    \begin{tabular}{ |c|c|c| } 
        \hline
        Level & Original ADT & Promoted Kind \\
        \hline 
        Kinds &  & Nat \\
        Types & Nat & 'Zero, 'Succ \\
        Values & Zero, Succ & \\
        \hline
    \end{tabular}
    \caption{Illustration of the promoted kinds automatically created by \code{DataKinds}}
    \label{fig:datakindsPromotion}
\end{figure}

%TODO: Motivation for GADT, well typed by contruction, typesafe (pre-condition), problem-> Solution, ADT Implementation and drawbacks

We use the \code{'Zero} and \code{'Succ} type constructors to represent numbers at the type level. Using the \code{GADTs} extension we now define the datatype for length-indexed vectors using GADT syntax.

\begin{lstlisting}
    data Vec :: * -> Nat -> * where
        ZeroVec :: Vec a 'Zero
        SuccVec :: a -> Vec a n -> Vec a ('Succ n)
\end{lstlisting}
\cite{GADTs}

% TODO: PolyKinds vs KindSignatures

A value of type \code{Vec a n} is a vector of \code{n} elements of type \code{a}. For this definition we also require the \code{KindSignatures} compiler extension to specify the kind signature of \code{Vec} in the first line of the definition. This specifies that the first type parameter of \code{Vec}, denoting the type of the elements of the vector, should be of kind \code{*}. This is because the type of the elements of the vector could be any concrete type, all of which are inhabitants of \code{*}. The first line also specifies that the second type parameter of \code{Vec}, denoting the length of the vector, should be a type inhabiting the promoted kind \code{Nat}, which we use to represent a type-level number. The returned kind \code{*} in the kind signature specifies that each vector has a concrete type of kind \code{*}.
% TODO: Define "concrete type"

Since \code{Vec} is a GADT, the types of each constructor are given explicitly. The \code{ZeroVec} constructor creates a vector with elements of any type (since \code{a} is universally quantified over) of length \code{'Zero}. The \code{SuccVec} constructor takes a value of type \code{a} and a vector of \code{n} elements of type \code{a}, and returns a new vector of length \code{'Succ n}. For example, the vector \code{SuccVec "a" (SuccVec "b" ZeroVec)} has type \code{Vec String ('Succ ('Succ 'Zero))}.

An immediate advantage of using GADT length-indexed vectors over standard lists is that we can define a new function \code{head'} which only operates on vectors containing at least one element. 

\begin{lstlisting}
    head' :: Vec a (Succ n) -> a
    head' (SuccVec x xs) = x
\end{lstlisting}

The additional type precision awarded by GADTs moves errors from run-time to compile-time.

% TODO: Additional precision with types resused many time in this project

\subsection{The Elem GADT}

Before using GADTs to construct the set of well-typed, we first define the \code{Elem} GADT. A value of type \code{Elem xs x} is a proof that the type \code{x} is an element of the list of types \code{xs}. 

\begin{lstlisting}
    data Elem :: [a] -> a -> * where
        Head :: Elem (x ': xs) x
        Tail :: Elem xs x -> Elem (y ': xs) x
\end{lstlisting}
\cite{GADTs}

The \code{Head} constructor produces a proof that \code{x} is an element of any list beginning with \code{x}. Given a value of type \code{Elem xs x}, the \code{Tail} constructor produces a proof that \code{x} is an element of the extended list \code{y:xs} for any element \code{y}. For example, the value \code{Tail Head} could have type \code{Elem '["a","b","c","d"] "b"} or type \code{Elem '[4, 5] 5}. 

Note that all elements of said lists are at the type level, so we need a promoted type-level version of the \code{(:)} list constructor. Promotion to the type-level constructor \code{'(:)} is handled by \code{DataKinds}, however we need to enable the \code{TypeOperators} extension to allow the use of the infix operators at type-level. Without the \code{TypeOperators} extension we could use the syntax \code{Elem ('(:) x xs)} in the definition of \code{Head}, however this is harder to read.

Note that \code{[a]} is the kind consisting of lists with type-level elements of the same kind \code{a}, rather than the standard list of value-level elements of the same type. To write this polymorphic kind signature we need the \code{PolyKinds} extension, which extends the \code{KindSignatures} extension by enabling polymorphic kind signatures. Since \code{KindSignatures} is a dependency of \code{PolyKinds} it is implicitly enabled when using \code{PolyKinds}, so we can replace the \code{KindSignatures} extension declaration with \code{PolyKinds}.

% TODO: Curry-Howard Isomorphism

\subsection{The Expr GADT}

We are now ready to define the \code{Expr} GADT which represents the set of well-typed terms of the STLC.

\begin{lstlisting}
    data Expr :: [Ty] -> Ty -> * where
        Var :: Elem ctx ty -> Expr ctx ty
        Lam :: Expr (arg ': ctx) result -> Expr ctx (arg ':-> result)
        App :: Expr ctx (arg ':-> result) -> Expr ctx arg -> Expr ctx result 
\end{lstlisting}

A value of type \code{Expr ctx ty} is a well-typed expression of the STLC of type \code{ty} in the typing context \code{ctx}. Hence, \code{Expr} encodes the set of valid typing judgements, where each constructor encodes a typing judgement rule in its type.
% TODO: explain typing context
% TODO: DeBruijn Variables

The \code{Var} constructor corresponds to the variable typing judgement rule. A variable can only be a well-typed expression of type \code{ty} if it is present with the correct type in the typing context. \code{Var} takes a value of type \code{Elem ctx ty} which proves that \code{ty} is in the typing context \code{ctx} by specifying which element of the context the variable is referring to.
% TODO: Check above
Note that instead of indexing variables by names or de Bruijn indices as we saw in Chapter \ref{chap:untypednbe}, typed variables are indexed by an \code{Elem} value. However, the values of type \code{Elem xs x} act very similarly to de Bruijn indices, as \code{Elem} values refer to bound variables by how many bindings are between the instance of the variable and where it was bound. For example \code{Var (Tail Head)} could have type \code{Expr '[BaseTy, BaseTy :-> BaseTy, BaseTy] ('BaseTy :-> 'BaseTy)}, where \code{Tail Head} has the type \code{Elem '[BaseTy, BaseTy :-> BaseTy, BaseTy] ('BaseTy :-> 'BaseTy)}. The \code{Elem} value \code{Tail Head} tells us where to find the type in the typing context, which in turn specifies how many new bound variables have been introduced since the variable we are interested in was bound, exactly like a de Bruijn index.

The \code{Lam} constructor corresponds to the abstraction typing judgement rule. Given a well-typed expression of type \code{result} in the context \code{arg:ctx}, we can abstract out the first variable of context into a bound variable with a lambda expression, producing a new term with the function type \code{arg :-> result} in the weakened context. We refer to the argument of the \code{Lam} constructor as the body of the lambda.

The \code{App} constructor corresponds directly to the application typing judgement rule, where we can apply one term to another if they share the same context \code{ctx} and the type of the second term matches the argument type \code{arg} of the first term.

% TODO: Why awesome?
This syntax offers multiple allows 

% TODO: Input/Ouput type as pre-condition/post-condition (eg empty context), programs as proof

Note that the apostrophes on type-level constructors are not required for successful compilation, as GHC can infer whether the constructor is type-level or value-level automatically. 
% TODO: Useful for getting started, now bulky, omit for remainer of dissertation

\section{Notes}

Advantage over ADTs: type refinement by constructor

Example 3: Expr - all values well typed by construction

Input/Ouput type as pre-condition/post-condition (eg empty context)


% -----------------------------------------------------------------------------

\chapter{Normalising the Typed Lambda Calculus}
\label{chap:typednbe}

\section{Target Syntax}

% TODO: Colour for polymorphic type variable

\begin{lstlisting}
    data NormalExpr :: [Ty] -> Ty -> * where
        NormalNeutral :: NeutralExpr ctx ty -> NormalExpr ctx ty
        NormalLam :: NormalExpr (arg ': ctx) result -> NormalExpr ctx (arg :-> result)    

    data NeutralExpr :: [Ty] -> Ty -> * where
        NeutralVar :: Elem ctx ty -> NeutralExpr ctx ty
        NeutralApp :: NeutralExpr ctx (arg :-> result) -> NormalExpr ctx arg 
                   -> NeutralExpr ctx result

\end{lstlisting}

% TODO: Justify why V has context (need to reify, example, normalise identity), describe what we need

\section{Evaluation Constraints}

% TODO: First evaluation problem - returning correct semantic context, lambda has different syntax
% - Solved for us by Andras

\begin{lstlisting}
    
\end{lstlisting}
% TODO: Correspondence with untyped env, Motivate need for tracking type, ctxV

\section{Order Preserving Embeddings (OPEs)}

\begin{lstlisting}
    data OPE :: [Ty] -> [Ty] -> * where
        Empty :: OPE '[] '[]
        Drop  :: OPE ctx1 ctx2 -> OPE (x : ctx1) ctx2
        Keep  :: OPE ctx1 ctx2 -> OPE (x : ctx1) (x : ctx2)
\end{lstlisting}

% TODO: Motivate why need in terms of eval, read Andras Paper

\section{Semantic Set}

\begin{lstlisting}
    data V :: [Ty] -> Ty -> * where 
        Base :: NormalExpr ctx BaseTy -> V ctx BaseTy
        Function :: (forall strong . OPE strong weak -> V strong arg -> V strong result) 
                 -> V weak (arg :-> result)
\end{lstlisting}

% TODO: Rank-2 extention, OPE relevance

\section{Evaluation}

\subsection{Environment}

\begin{lstlisting}
    data Env :: [Ty] -> [Ty] -> * where
        EmptyEnv :: Env '[] ctxV
        ConsEnv  :: Env ctx ctxV -> V ctxV ty -> Env (ty : ctx) ctxV
\end{lstlisting}

\begin{lstlisting}
    envLookup :: Elem ctx ty -> Env ctx ctxV -> V ctxV ty 
    envLookup Head     (ConsEnv _    v) = v
    envLookup (Tail n) (ConsEnv prev _) = envLookup n prev
\end{lstlisting}

\subsection{Variable Case}

\begin{lstlisting}
    eval :: Env ctx ctxV -> Expr ctx ty -> V ctxV ty
    eval env (Var n) = envLookup n env
\end{lstlisting}

\subsection{Lambda Case}

\begin{lstlisting}
    eval (env :: Env ctx ctxV) (Lam (body :: Expr (arg:ctx) result)) = Function f 
        where
            f :: (SingContext ctxV') => OPE ctxV' ctxV -> V ctxV' arg -> V ctxV' result
            f ope v = eval (ConsEnv (strengthenEnv ope env) v) body
\end{lstlisting}

%TODO: Scoped Type variables
% - Show what types are inferred without

% TODO: Remove singletons

\begin{lstlisting}
    strengthenElem :: OPE strong weak -> Elem weak ty -> Elem strong ty
    strengthenElem Empty      v        = v
    strengthenElem (Drop ope) v        = Tail (strengthenElem ope v)
    strengthenElem (Keep ope) Head     = Head
    strengthenElem (Keep ope) (Tail v) = Tail (strengthenElem ope v)

    strengthenNormal :: OPE strong weak -> NormalExpr weak ty -> NormalExpr strong ty
    strengthenNormal ope (NormalNeutral n) = NormalNeutral (strengthenNeutral ope n)
    strengthenNormal ope (NormalLam n)     = NormalLam (strengthenNormal (Keep ope) n)

    strengthenNeutral :: OPE strong weak -> NeutralExpr weak ty -> NeutralExpr strong ty
    strengthenNeutral ope (NeutralVar n)   = NeutralVar (strengthenElem ope n)
    strengthenNeutral ope (NeutralApp f a) = NeutralApp (strengthenNeutral ope f) (strengthenNormal ope a) 

    strengthenV :: (SingContext strong) => OPE strong weak -> V weak ty -> V strong ty
    strengthenV ope                      (Base nf) = Base (strengthenNormal ope nf)
    strengthenV (ope :: OPE strong weak) (Function (f :: forall strong . (SingContext strong) => OPE strong weak -> V strong arg -> V strong result)) = Function f' 
        where
            f' :: (SingContext stronger) => OPE stronger strong -> V stronger arg -> V stronger result
            f' ope' = f (composeOPEs ope ope')
    
    strengthenEnv :: (SingContext c) => OPE c b -> Env a b -> Env a c
    strengthenEnv _   EmptyEnv         = EmptyEnv
    strengthenEnv ope (ConsEnv tail v) = ConsEnv (strengthenEnv ope tail) (strengthenV ope v)
\end{lstlisting}

\subsection{Application Case}

%TODO: Motivate by pattern matching on type not possible in Haskell

\begin{lstlisting}
    eval env (App f a) = appV (eval env f) (eval env a) 
        where
            appV (Function f') a' = f' (idOPEFromEnv env) a'

            idOPEFromEnv :: (SingContext ctxV) => Env ctx ctxV -> OPE ctxV ctxV
            idOPEFromEnv _ = idOpe 
\end{lstlisting}

\begin{lstlisting}
    class SingContext ctx where
        idOpe :: OPE ctx ctx

    instance SingContext '[] where
        idOpe = Empty

    instance (SingContext xs, SingTy x) => SingContext (x:xs) where
        idOpe = Keep idOpe
\end{lstlisting}

%TODO Include code link

\section{Reification}

\begin{lstlisting}
    reify :: V ctx ty -> NormalExpr ctx ty
    reify (Base nf)    = nf
    reify (Function f) = NormalLam (reify (f ope (evalNeutral (NeutralVar Head)))) 
        where
            ope = weakenContext (Function f)

            weakenContext :: (SingContext ctx) => V ctx ty -> OPE (x:ctx) ctx
            weakenContext _ = wk 

    evalNeutral :: (SingTy ty, SingContext ctx) => NeutralExpr ctx ty -> V ctx ty
    evalNeutral = evalNeutral' singTy

    evalNeutral' :: (SingContext ctx) => STy ty -> NeutralExpr ctx ty -> V ctx ty
    evalNeutral' SBaseTy       n                                       = Base (NormalNeutral n)  
    evalNeutral' (SArrow _ _)  (n :: NeutralExpr ctx (arg :-> result)) = Function f 
        where
            f :: (SingContext strongerCtx) => OPE strongerCtx ctx -> V strongerCtx arg -> V strongerCtx result
            f ope v = evalNeutral (NeutralApp (strengthenNeutral ope n) (reify v))
\end{lstlisting}

% TODO: Motivate evalNeutral and rename, contrast with standard eval

\begin{lstlisting}
    data STy :: Ty -> * where
        SBaseTy :: STy BaseTy
        SArrow  :: (SingTy a, SingTy b) => STy a -> STy b -> STy (a :-> b)

    class SingTy a where
        singTy :: STy a 

    instance SingTy 'BaseTy where
        singTy = SBaseTy

    instance (SingTy a, SingTy b) => SingTy (a :-> b) where
        singTy = SArrow singTy singTy
\end{lstlisting}

\begin{lstlisting}
    class SingContext ctx where
        idOpe :: OPE ctx ctx
        wk :: OPE (x:ctx) ctx
        wk = Drop idOpe
\end{lstlisting}

\section{Normalisation}

\begin{lstlisting}
    normalise :: (SingContext ctx) => Expr ctx ty -> NormalExpr ctx ty
    normalise = reify . eval initialEnv
\end{lstlisting}

\begin{lstlisting}
    class SingContext ctx where
        idOpe :: OPE ctx ctx
        wk :: OPE (x:ctx) ctx
        wk = Drop idOpe
        initialEnv :: Env ctx ctx

    instance SingContext '[] where
        idOpe = Empty
        initialEnv = EmptyEnv

    instance (SingContext xs, SingTy x) => SingContext (x:xs) where
        idOpe = Keep idOpe
        initialEnv = ConsEnv (strengthenEnv wk initialEnv) (evalNeutral (NeutralVar Head))
\end{lstlisting}

\section{Notes}

Investigation: Are GADTs in Haskell powerful enough? Types are erased at runtime so true dependent typing not part of Haskell (programs at type level)

Advantage over ADTs: type refinement by constructor

Poissible to erase all type information, NbE on Untyped
Issue: No proof that type preserved 
Solution: Track types as do evaluation - nbe program itself proof that types preserved (subject reduction parallel?)

Started by implementing same as untyped

Main difference in semantics (V := a -> b | Neutral) 
\cite{slides}

problem: Need to strengthen context evaluating body (eval Lam case)
\subsection{Solution: Order Preserving Embeddings (OPEs)}

Following implementation in Agda \cite{AgdaNbe}, agda has full dependent types (type system more powerful) - adapt for haskell, how nicely? 

if a term well typed for one context, also well typed for any longer one

A value of type 'OPE strong weak' can derive weak from strong by dropping elements from context

OPE is a relation on typing contexts

\subsection{Semantic set}

Defintion of V using OPEs - Haskell vs agda

Need to quantify over 'strong' in function - OPE strong weak is guarentee that strong is a stronger context than weak (if quantified at start end up with values where weak stronger than strong) - need rank2 types extension for nested quantification

Helper functions (composition, strengthing relative to OPE) - explain derivations

\subsection{implementing Eval}

Defintion of environment (maps expressions in syntax context ctx to values in semantics with context ctxV)

problem: in app case how to we get identity OPE for semantic context?

But types erased at compile time to make Haskell efficient

How to generate a value at runtime dependent on type erased at compile time

dependent pattern match \cite{SingletonsGuide}

\subsection{Solution: Singleton pattern}

Method of Type to value known as reflection \cite{SingletonsGuide}

Idea: Create value-level tags for types - singleton types correspond type we're interested in, inhabited by only one value for each case

Examples: Reify case analysis, Ty reflection, Context reflection

Explicitly passing as value to pattern match on

Generate implictly using typeclass, use class constraint to imiplictly pass down ability to use contex methods through function calls.
Is it a good idea to have class constraints in the GADTs/Syntax definitions?

Implementation in class vs full reflection - test this for speed?

problem : Inferring Any for ctxV (why?)

solution: scoped type variables - universally quantified variables used in type expressions bind over 'where' clause

(More usefully) can 'unpack' refined GADT types so that can create type definitions using refined types.

Analysis:

Have to specify type when normaling for correct eta-expansion (eta-long form)

Qs:
How does locally nameless work in sematics?
How does ctxV work in Env?


% -----------------------------------------------------------------------------

\chapter{Critical Evaluation}
\label{chap:evaluation}

% This chapter is intended to evaluate what you did.  The content is highly 
% topic-specific, but for many projects will have flavours of the following:
% 
% \begin{enumerate}
% \item functional  testing, including analysis and explanation of failure 
%       cases,
% \item behavioural testing, often including analysis of any results that 
%       draw some form of conclusion wrt. the aims and objectives,
%       and
% \item evaluation of options and decisions within the project, and/or a
%       comparison with alternatives.
% \end{enumerate}
% 
% \noindent
% This chapter often acts to differentiate project quality: even if the work
% completed is of a high technical quality, critical yet objective evaluation 
% and comparison of the outcomes is crucial.  In essence, the reader wants to
% learn something, so the worst examples amount to simple statements of fact 
% (e.g., ``graph X shows the result is Y''); the best examples are analytical 
% and exploratory (e.g., ``graph X shows the result is Y, which means Z; this 
% contradicts [1], which may be because I use a different assumption'').  As 
% such, both positive {\em and} negative outcomes are valid {\em if} presented 
% in a suitable manner.

\section{Testing}

\section{Evaluation of Language Features}

\subsection{GADTs}

We found that \code{GADTs} is an excellent extension for emulating dependent types when combined with \code{PolyKinds}. GADT syntax allowed us to define types \code{Expr} with much greater precision and enabled us to translate Agda \code{data} definitions such as \code{OPE} and type generating functions such as $\text{Ty}^\text{N}$, that otherwise could not be expressed in Haskell.

However, during the analysis and testing of our implementation, we found that GADTs indexed by types can create difficulties for deriving useful class instances. For example, it would be desirable for \code{NormalForm} to be a member of the \code{Eq} class, so that during testing we could directly check if two normal forms are equal. 

\begin{lstlisting}
    instance Eq (NeutralForm ctx ty) where
        (NeutralVar n) == (NeutralVar m) = n == m
        (==) (NeutralApp (n :: NeutralForm ctx (arg1 :-> ty)) m) 
             (NeutralApp (x :: NeutralForm ctx (arg2 :-> ty)) y) 
              = n == x && m == y 
        _ == _ = False
\end{lstlisting}

Here we have attempted to implement an \code{Eq} instance for \code{NeutralForm}, which is needed to determine equality between normal forms. The variable case doesn't pose any problems, as GHC can automatically derive an \code{Eq} instance for \code{Elem}. However, in the application case we cannot be sure that \code{arg1} and \code{arg2} are equal, since they are universally quantified type variables. Because of this, \code{n} and \code{x} may have different types, and so we cannot determine equality between them. Thus, the instance fails to compile.

We attempted to benchmark the typed NbE implementation, to evaluate how much performance overhead the additional type information incurred compared to the untyped implementation. However, to use Haskell's benchmarking features, we need a \code{NormalForm} instance for the \code{NFData} class. As \code{NormalForm} is a GADT, GHC is unable to automatically derive the instance. This further exemplifies how the class limitation makes GADTs more difficult to work with, and it would be interesting to explore whether this limitation could be overcome.

In general, we believe that GADTs are a useful tool for Haskell developers to improve the type-safety of their programs by taking advantage of the programs as proof paradigm. However, the instance deriving limitation shows that the extension is not completely mature yet.

\subsection{Dependent Pattern Matching}

For our implementation, the class instance trick for performing dependent pattern matching on types worked well. Passing singleton class constraints through the program did add syntactic noise and required some thought, but passing constraints through type signatures had no effect on the function implementations. In some situations we were forced to add type constraints to data types themselves. 
% TODO: Needs type annotation, implementation has effect on data, not nice.

Our boilerplate overhead was small, since we only needed dependent pattern matches for a few functions. However, as noted in Section \ref{sect:typedNormalisation} for more complex applications with many dependent pattern matches, this boilerplate could become more difficult to manage. The boilerplate was also compact since our implementation only required pattern matches on the structure of types. For example, \code{idOpe} only depended on the length of the type-level context, so there was no need to identify its contents. This meant we only used a portion of the singleton pattern, which can be used to extract all the type-level information into the value level, at the cost of additional boilerplate.

Richard Eisenberg's proposals for built-in dependent pattern matches using a syntax to Agda would be a much cleaner approach than \code{GADTs}, but the current solution for managing dependent pattern matches is the \code{singleton} library. This library uses template Haskell to automatically generate the instances and data types needed for “faking” dependent pattern matches, and would be useful for dependent pattern matching at scale.

\subsection{Arbitrary-Rank Polymorphism}

When defining the \code{Function} constructor of the STLC semantic set, we used explicit \code{forall} syntax to implement nested polymorphism. Whilst this satisfied most type requirements, when constructing concrete \code{Function} elements, GHC could not infer necessary type variable equalities to complete type-checking. We resolved this issue by using type annotations to bind type variables in function arguments. 

An interesting area for further study could be to identify why type inference was not able to derive the type variable equalities in these cases. We suspect that it could be due to the higher ranked polymorphism, as \code{Function} was the only definition in our implementation that made use of the \code{RankNTypes} extension.



% -----------------------------------------------------------------------------

\chapter{Conclusion}
\label{chap:conclusion}

% {\bf A compulsory chapter,     of roughly $5$ pages} 

% \noindent
% The concluding chapter of a dissertation is often underutilised because it 
% is too often left too close to the deadline: it is important to allocation
% enough attention.  Ideally, the chapter will consist of three parts:
% 
% \begin{enumerate}
% \item (Re)summarise the main contributions and achievements, in essence
%       summing up the content.
% \item Clearly state the current project status (e.g., ``X is working, Y 
%       is not'') and evaluate what has been achieved with respect to the 
%       initial aims and objectives (e.g., ``I completed aim X outlined 
%       previously, the evidence for this is within Chapter Y'').  There 
%       is no problem including aims which were not completed, but it is 
%       important to evaluate and/or justify why this is the case.
% \item Outline any open problems or future plans.  Rather than treat this
%       only as an exercise in what you {\em could} have done given more 
%       time, try to focus on any unexplored options or interesting outcomes
%       (e.g., ``my experiment for X gave counter-intuitive results, this 
%       could be because Y and would form an interesting area for further 
%       study'' or ``users found feature Z of my software difficult to use,
%       which is obvious in hindsight but not during at design stage; to 
%       resolve this, I could clearly apply the technique of Smith [7]'').
% \end{enumerate}

% Summary of what we did 
%   3 variations
%   possible to translate Agda implementation to Haskell with type guarentees

\section{Summary}
In this dissertation we produced three variations of NbE. The first normalised the untyped lambda calculus using the Gensym approach, following an implementation by Lindley \cite{slides}. Issues with fresh variable names inspired a second implementation for untyped de Bruijn terms which we adapted from a specification by Abel \cite{deBruijn}. GADTs were used to define the STLC in Haskell, before successfully translating an implementation of NbE for the SLTC by Kovacs \cite{AgdaNbe} from Agda to Haskell. In the evaluation, we verified that that our implementations are correct, and found that GHC compiler extensions are powerful enough to emulate some features of dependently typed languages, albeit with practical drawbacks to their use.

\section{Evaluation of Aims}

We met both of our original aims, although there are areas for further study in both.

Our first aim was to produce various implementations of NbE in Haskell, which we achieved in Chapters \ref{chap:untypednbe} and \ref{chap:typednbe} and verified in Section \ref{section:testing}. This aim was intentionally left very open, since this was an exploratory project where it was difficult to predict the problems that would arise, and the amount of time each implementation would take. Each implementation served a valuable purpose. The Gensym implementation helped us understand the operation of NbE. The de Bruijn implementation was useful in deepening this understanding, which we evidenced by adding our own variations on the guiding implementation. The typed implementation allowed us to test advanced typing features, and presented challenges when Haskell fell short of Agda's features.

Our second aims was to explore how successful modern features of Haskell are in implementing an algorithm with complex types, which we discussed in Chapter \ref{chap:evaluation}. We saw that all the advanced features worked well at achieveing their specific goals, and made it possible to construct programs that would not be possible in standard Haskell. However, the recurring conclusion was that these features incur a significant amount of additional work in practice, and can lead to difficult problems. Thus, their use is hard to justify unless the type-security of the program is the priority. The release of Dependent Haskell \cite{DH} should address some of these issues by making dependent features a more central part of the language.

\section{Areas for Further Study}

As part of our second aim, we would have liked to benchmark NbE for the STLC against the untyped implementations to evaluate the performance trade-off of the additional type information. As mentioned in Section \ref{subsect:GADTanalysis}, we were unable to construct the necessary class instances to benchmark our normalisation, however this is an interesting problem that others are already researching \cite{gadtClassInstances}. If solved, it would also be interesting to benchmark the Haskell implementation of typed NbE against Kovac's Agda implementation, as a significant difference in performance could be compelling reason to choose one over the other in practice.

In Section \ref{subsect:rankNTypes}, we suggested an investigation into whether the \code{RankNTypes} extension leads to overly general type inference.

In Section \ref{sect:typedNormalisation}, we identified the additional challenge of translating Kovacs' full proof of correctness for the typed NbE implementation from Agda to Haskell. If possible, we would have extremely strong evidence that the Haskell \code{normalise} implementation produces the correct normal form. 


\backmatter

\printbibliography

% -----------------------------------------------------------------------------

% The dissertation concludes with a set of (optional) appendicies; these are 
% the same as chapters in a sense, but once signaled as being appendicies via
% the associated macro, LaTeX manages them appropriatly.

\appendix

\chapter{An Example Appendix}
\label{appx:example}

Content which is not central to, but may enhance the dissertation can be 
included in one or more appendicies.

\noindent
Note that in line with most research conferences, the marking panel is not
obliged to read such appendices.

% =============================================================================

\end{document}
