\chapter{Introduction}
\label{chap:introduction}

% Start with want to implement functional language
% We use NbE
% What/how it works roughly
% Why do we care
    % Modern
    % Dependent type theories (only namecheck)
    % Tests limit of implementation lanague (Haskell)
    % Closer to maths
    % Conceptual understanding

% --roadmap
% What do we do in the project
    % "Explore various implementations"
    % NbE for untyped lc (2 variations, abel trick (dB indicies + levels))
    % NbE for simply typed LC ("Cutting edge features")
        % GADTs
        % Type families
        % Extentions list

% -- Why awesome
% Why we did it this way 
    % Haskell not depdently typed (GADTs close as can get)
    % R. Eisenberg + community plan for full dependent types
    % Andras Kovacs implementation in Agda + port to Haskell
        % point is to discuss problems and workarounds

% TODO: Introduce more foundations in logic

%   Intro to Dependent types
Dependent types are types that can depenend on values. For example, the return type of a dependently typed function can depend on the value of its argument. This feature is useful as it allowers developers to construct more presise and expressive definitions, moving errors from run-time to compile-time.

% Type checking depent types
Naturally, this dependence on run-time values increases the complexity of static type-checking. 
Since the types themselves may contain programs, for the compiler to validate equality between types, it needs to effficiently validate equality between programs at the type level.
% Q: How to programs arise from a dependence on values?
% TODO: Rephrase this (says LC twice)
In functional programming lanaguages, programs are terms of the Lambda Calculus, so the problem reduces to determinig equality between Lambda Calculus terms. 


We say that two terms are $\beta$-convertible if their normal forms are syntactically identical (providing they exist). Normal forms are the simplest syntactic representation of lambda calculus terms which preserve their semantic meaning. All terms with the same normal form will produce identical outputs for all inputs when interpreted as functions, so are semantically equal. To deduce equality between type-level terms we can simply compare their $\beta$-normals forms.

However, the standard syntactic-rewriting method of computing normal forms (normalisation by reduction) fails to satisfy desireable properies we would like a dependent type-checker to have:
\begin{itemize}
    % TODO: Rephrase this
    \item 
    It would be most natural for our definition of equality to be undirected, however normalistion by reduction is a directed technique.

    %   eta-long form not directed 
    % Q: What is desire for eta-long form?
    \item We would like to normlise terms into eta-long form, however ... 
    
    %   Not efficient??
    \item Performance is always a priority for compilers so more effficient normalisation would be preferable.
    
    %   Rewriting difficult to reason about mathematically (operating on syntax)
    \item The ability to mathematically reason about our system is essential for ensuring correctness. Normalisation by reduction is based on syntax operations, which makes such rigorous reasoning difficult.
\end{itemize}

% Resolution: NbE
Instead we consider an alternative method of normalising lamda terms: normalisation by evaluation. NbE proceeds by interpreting the term as an element of a semantic set, before "reifying" the semantic value back into the set of normal lambda terms. 

% TODO:rephrase
This technique is undirected ...
% TODO:rephrase
The  $\beta$-normal forms that are also $\eta$-long
Crucially, the technique is built from mathematics, using mathematical operations rather than syntactic ones, so is enforced by theory.

% What is this project about?

%   Modern compiler extentions support more complex programs at type level
In this dissertation, we construct and analyse varying implementations of NbE in Haskell, a lazy functional programming lanaguage. Haskell is the most widely adopted pure functional programming lanaguage, however was not intended to support complex dependent types. In fact, to support compiler optimisations Haskell erases all types from the program after type-checking, so at run-time all type information is lost. To add stronger typing features, the developers of Haskell have realeased mordern compiler extentions. 

In this project we test these modern, advanced features through implementating vartations of NbE for both the untyped and typed Lambda Calculus. 

The aims of this dissertation are to:
\begin{itemize}
    \item Implement normalisation by evaluation for the typed and untyped Lambda Calculus in Haskell
    
    % TODO: Rephase this
    \item Investigate modern typing features of Haskell, particularly how useful they are in developing dependently typed programs and how the implementation compares to dependently typed languages.
\end{itemize}
